\documentclass[12pt]{book}   
\usepackage{graphics}
\usepackage{spaccent}  
\usepackage[spanish]{babel} 
\usepackage{fancybox, calc}
\newcommand {\?}{?`}  
\newcommand{\B}{\mathbb{B}}
\newcommand{\A}{\mathbb{A}}
\newcommand{\D}{\mathbb{D}}
\newcommand{\U}{\mathbb{U}}
\newcommand{\M}{\mathbb{M}}
\newcommand{\J}{\mathbb{J}}
\newcommand{\K}{\mathbb{K}}
\newcommand{\X}{\mathbb{X}}
\newcommand{\C}{\mathbb{C}}
\newcommand{\I}{\mathbb{I}}

\begin{document}
\specialaccent \mbox{} 
\begin{center}
{\Large \bf Tarea 1 (B) Reconocimiento de Patrones} \mbox{} \vspace{0.5cm} \\
\end{center}




\begin{enumerate}

\item (no hay que entregar nada) A aquellos que se sienten aun no muy familiarizados con an\'alisis de datos,
recomiendo leer la parte del libro {\tt Applied Multivariate Analysis}  sobre  un an\'alisis de PCA de los datos de (otro) heptatlon  a partir de pag. 78 (pag.  92 en el pdf). \\
Nota: en el libro se usa {\tt prcomp} y no {\tt princomp}. Ambos calculan PCA; la diferencia es m\'as bien en el m\'etodo n\'umerico subyacente que se usa:  {\tt prcomp()} usa SVD y {\tt princomp()} usa la matriz de covarianza.
En general se considera  que desde punto num\'erico {\tt prcomp()} es mejor pero es m\'as dificil sacar las proyecciones y scores. Si \verb|objeto<- prcomp()|, entonces  \verb|objeto$rotation[,1]|  es el equivalente a lo que da loadings[,1] con  {\tt princomp()} (en Python por default se va por SVD).




\item (usaremos  este resultado en la siguiente clase)\\
Sea $\{x_i\}$ un conjunto de $n$ vectores $d$ dimensional.  Definimos las matrices  $n \times n$  $[\K_{i,j}]$ con $\K_{i,j}= \langle x_i, x_j \rangle$ y $\D^2$ la matriz de distancias al cuadrada correspondiente: $\D^2_{i,j}=||x_i-x_j||^2$

Verifica la identidad:
\[ \D^2 = c 1^t + 1 c^t -2 \X\X^t , \]
con $1$ un vector de unos de longitud $n$ y $c$ el vector de longitud $n$ con elementos $( \K_{i,i})_{i=1}^n$

Hint: escribe primero $||x_i-x_j||^2$   en t\'erminos de productos puntos.

\item
Acerca de la demostraci\'on de la maximizaci\'on del cociente de Rayleigh ($\max_l \frac{l^tCov(X)l}{l^t l}$):
Haz unos peque\~nos cambios necesarios para demostrar que el segundo vector propio de $Cov(X)$ es la soluci\'on del problema de maximizar el cociente bajo la restricci\'on adicional de ser ortogonal al primer vector propio.  
(por si sirve: aqu\'i una grabaci\'on vieja de la demostraci\'on en tiempos de la pandemia {\tt https://www.youtube.com/watch?v=8TBpSUXcDww })




\item
Considera los datos {\it oef2.data}. Se trata de los promedios mensuales de la temperatura (en Celsius) en 35 estaciones canadienses de monitoreo. El inter'es es
comparar las estaciones entre s'i en base de sus curvas de temperatura.
Consideramos  las 12 mediciones por estaci'on como las entradas de un vector (v.a.) $X$



\begin{enumerate}
\item
Busca algunas visualizaciones informativos de los datos.
\item 
Calcula los componentes principales.
\item 
Aprovechando que las columnas hacen referencia a meses consecutivos, tiene sentido dibujar cada componente principal como una gr\'afica. Por ejemplo graficar $\{(i,{l_1}_i)\}$ y $\{(i,{l_2}_i)\}$ con $i$ de 1 a 12 y $l_1$ y $l_2$ el primer y segundo componente principal. \\



\? Qu\'e interpretaci\' on das al primer y segundo componente?



\end{enumerate}


Para leer los datos en {\cal R}
\begin{verbatim}
temp <- matrix(scan("oef2.data"), 35, 12, byrow=T)

nombresestaciones <-   c("St. John_s",    "Charlottetown", "Halifax" ,
                  "Sydney",        "Yarmouth",      "Fredericton",
                  "Arvida",        "Montreal",      "Quebec City",
                  "Schefferville", "Sherbrooke",    "Kapuskasing",
                  "London",        "Ottawa",        "Thunder Bay",
                  "Toronto",       "Churchill",     "The Pas",
                  "Winnipeg",      "Prince Albert", "Regina",
                  "Beaverlodge",   "Calgary",       "Edmonton",
                  "Kamloops",      "Prince George", "Prince Rupert",
                  "Vancouver",     "Victoria",      "Dawson",
                  "Whitehorse",    "Frobisher Bay", "Inuvik",
                  "Resolute",      "Yellowknife")

rownames(temp)<-nombresestaciones

\end{verbatim} \mbox{} \\




\end{enumerate}


\end{document}
