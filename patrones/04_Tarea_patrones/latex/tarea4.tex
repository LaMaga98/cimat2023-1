%%%%%%%%%%%%%%%%%%%%%%%%%%%%%%%%%%%%%%%%%%%%%%%%%%%%%%%%%%%%%%%%%%%%%%%%%%%%
%%%%%%%%%%%%%%%%%%%%%%%%%%%%%%%%%%%%%%%%%%%%%%%%%%%%%%%%%%%%%%%%%%%%%%%%%%%%
%%%%%%%%%%%%%%%%%%%%%%%%% PAQUETES QUE UTILIZO %%%%%%%%%%%%%%%%%%%%%%%%%%%%%
%%%%%%%%%%%%%%%%%%%%%%%%%%%%%%%%%%%%%%%%%%%%%%%%%%%%%%%%%%%%%%%%%%%%%%%%%%%%
%%%%%%%%%%%%%%%%%%%%%%%%%%%%%%%%%%%%%%%%%%%%%%%%%%%%%%%%%%%%%%%%%%%%%%%%%%%%

\documentclass[letter, 11pt, twoside]{report}
\usepackage{amsthm}
\usepackage[many]{tcolorbox}
\usepackage{thmtools}
\usepackage{amssymb,bm,amsfonts,amsmath}
\usepackage[utf8]{inputenc}
\usepackage[spanish]{babel}
\usepackage[export]{adjustbox}
\usepackage{hyperref}
\usepackage{enumerate}
\usepackage{makeidx}
\usepackage{float}
\usepackage{graphicx, import}
\usepackage{subfig}
\usepackage{upgreek}
\usepackage{float}
\usepackage[all]{xy}
\usepackage{thmtools}
\usepackage{titlesec}
\usepackage{mathrsfs}
\usepackage{multicol}
\usepackage{tikz-cd}
\usetikzlibrary{patterns}
\usetikzlibrary{plotmarks}
\usepackage{wrapfig}
\usepackage{stmaryrd}
\usepackage{subfloat}
\usepackage{svg}
\usepackage{yfonts}
\usepackage{fancyhdr}
\usepackage{pifont}
\usepackage{pdfpages}
\usepackage{ marvosym }
\usepackage{hyperref}
\usepackage{pdflscape}
\usepackage{setspace}
\usepackage{color}
\usepackage{bm}
\usepackage{epigraph}
\usepackage{quotchap}
\usepackage[framemethod=TikZ]{mdframed}
\usepackage[nottoc,numbib]{tocbibind}
\usepackage[customcolors]{hf-tikz}
\usetikzlibrary{babel}
% PARA VER LAS REFERENCIAS LABELS
% \usepackage[notcite,color]{showkeys}
% CHECA http://www.tug.dk/FontCatalogue/iwonalightcondensed/
\usepackage[light,math]{iwona}
\usepackage[T1]{fontenc}
\usepackage{MnSymbol}
\usepackage{varwidth} % CAJA DE EJERCICIOS Y \SOMBREADO
\tcbuselibrary{vignette,many}
\tcbuselibrary{skins}
\usepackage{pgfplots}
\usepgfplotslibrary{fillbetween}
\pgfplotsset{compat=1.16}
\usepackage{xcolor}
% PARA ESCRIBIS CÓDIGO Y PSEUDOCÓDIGO
\usepackage{algorithm}
\usepackage{algpseudocode}
\usepackage{listings}
\usepackage{color, xcolor}


%%%%%%%%%%%%%%%%%%%%%%%%%%%%%%%%%%%%%%%%%%%%%%%%%%%%%%%%%%%%%%%%%%%%%%%%%%%%
%%%%%%%%%%%%%%%%%%%%%%%%%%%%%%%%%%%%%%%%%%%%%%%%%%%%%%%%%%%%%%%%%%%%%%%%%%%%
%%%%%%%%%%%%%%%%%%%%%%%%% MEMO PYTHON Y C %%%%%%%%%%%%%%%%%%%%%%%%%%%%%%%%%%
%%%%%%%%%%%%%%%%%%%%%%%%%%%%%%%%%%%%%%%%%%%%%%%%%%%%%%%%%%%%%%%%%%%%%%%%%%%%
%%%%%%%%%%%%%%%%%%%%%%%%%%%%%%%%%%%%%%%%%%%%%%%%%%%%%%%%%%%%%%%%%%%%%%%%%%%%

%CÓMO QUEDERÁ EL COLOREADO Y HIGHLIGHT DEL CÓDIGO
\definecolor{dkgreen}{rgb}{0.9,0.6,0.8}
\definecolor{blue}{rgb}{0.0,0.49,0.4}
\definecolor{gray97}{gray}{.97}
\definecolor{gray75}{gray}{.75}
\definecolor{gray45}{gray}{.45}
\definecolor{codepurple}{rgb}{0.58,0,0.82}
\definecolor{backcolour}{rgb}{0.95,0.95,0.92}
\definecolor{codegreen}{rgb}{0,0.6,0}
\definecolor{codegray}{rgb}{0.5,0.5,0.5}

\lstdefinestyle{mystyle}{
    backgroundcolor=\color{gray97},
    commentstyle=\color{cyan!75!black},
    keywordstyle=\color{magenta},
    numberstyle=\tiny\color{codegray},
    stringstyle=\color{codepurple},
    basicstyle=\ttfamily\footnotesize,
    breakatwhitespace=false,
    breaklines= true,
    captionpos=b,
    keepspaces=true,
    numbers=left,
    numbersep=5pt,
    showspaces=false,
    showstringspaces=false,
    showtabs=false,
    tabsize=2,
    language=bash,   %% PHP, C, Java, etc... bash is the standard
    extendedchars=true,
    inputencoding=latin1
}

\lstset{style=mystyle, literate =
                        {í}{{\'i}}1
                        {á}{{\'a}}1
                        {é}{{\'e}}1
                        {ó}{{\'o}}1
                        {ú}{{\'u}}1
                        {ñ}{{\~n}}1
                        {ü}{{\"u}}1
                            }

%%%%%%%%%%%%%%%%%%%%%%%%%%%%%%%%%%%%%%%%%%%%%%%%%%%%%%%%%%%%%%%%%%%%%%%%%%%%
%%%%%%%%%%%%%%%%%%%%%%%%%%%%%%%%%%%%%%%%%%%%%%%%%%%%%%%%%%%%%%%%%%%%%%%%%%%%
%%%%%%%%%%%%%%% COLORES DEL PAQUETE SHOWKEYS %%%%%%%%%%%%%%%%%%%%%%%%%%%%%%%
%%%%%%%%%%%%%%%%%%%%%%%%%%%%%%%%%%%%%%%%%%%%%%%%%%%%%%%%%%%%%%%%%%%%%%%%%%%%
%%%%%%%%%%%%%%%%%%%%%%%%%%%%%%%%%%%%%%%%%%%%%%%%%%%%%%%%%%%%%%%%%%%%%%%%%%%%

\definecolor{refkey}{rgb}{255,0,0}
\definecolor{labelkey}{rgb}{255,0,0}
\definecolor{mirosa}{HTML}{FF007F}

%%%%%%%%%%%%%%%%%%%%%%%%%%%%%%%%%%%%%%%%%%%%%%%%%%%%%%%%%%%%%%%%%%%%%%%%%%%%
%%%%%%%%%%%%%%%%%%%%%%%%%%%%%%%%%%%%%%%%%%%%%%%%%%%%%%%%%%%%%%%%%%%%%%%%%%%%
%%%%%%%%%%%%%%%% MARGENES, VIENE EN EL MANUAL DE LATEX %%%%%%%%%%%%%%%%%%%%%
%%%%%%%%%%%%%%%% FORMATO ME LO PASO RO %%%%%%%%%%%%%%%%%%%%%%%%%%%%%%%%%%%%%
%%%%%%%%%%%%%%%%%%%%%%%%%%%%%%%%%%%%%%%%%%%%%%%%%%%%%%%%%%%%%%%%%%%%%%%%%%%%
%%%%%%%%%%%%%%%%%%%%%%%%%%%%%%%%%%%%%%%%%%%%%%%%%%%%%%%%%%%%%%%%%%%%%%%%%%%%

\parskip=5pt
\hoffset = 0pt
\headsep = 1.5 cm % estaba en 1.5 cm, lo cambie para el header de la imagen
\oddsidemargin = .5cm
\evensidemargin = .5cm
\textheight = 657pt
\textwidth = 15.6cm
\topmargin = -2 cm
\parindent=0mm

%%%%%%%%%%%%%%%%%%%%%%%%%%%%%%%%%%%%%%%%%%%%%%%%%%%%%%%%%%%%%%%%%%%%%%%%%%%%
%%%%%%%%%%%%%%%%%%%%%%%%%%%%%%%%%%%%%%%%%%%%%%%%%%%%%%%%%%%%%%%%%%%%%%%%%%%%
%%%%%%%%%%%%%%%%%%%%%%%% CREACIÓN DE EJERCICIO %%%%%%%%%%%%%%%%%%%%%%%%%%%%%
%%%%%%%%%%%%%%%%%% MODIFICACIÓN PROOF Y QED %%%%%%%%%%%%%%%%%%%%%%%%%%%%%%%%
%%%%%%%%%%%%%%%%%%%%%%%%%%%%%%%%%%%%%%%%%%%%%%%%%%%%%%%%%%%%%%%%%%%%%%%%%%%%
%%%%%%%%%%%%%%%%%%%%%%%%%%%%%%%%%%%%%%%%%%%%%%%%%%%%%%%%%%%%%%%%%%%%%%%%%%%%

\renewcommand{\qedsymbol}{\tiny{$\blacksquare$}}

\newenvironment{solucion}{\begin{proof}[\textcolor{magenta}{Solución}]}{\end{proof}}

\newtcolorbox[auto counter]{ejercicio}[1][]{
% ESTO ES PARA LA CAJA GENERAL
breakable, % por si cambias de pagina
enhanced, % estilo general
% TITULO MODIFICACIONES
coltitle= black,
colbacktitle= white,
titlerule= 0mm,
colframe = magenta,
fonttitle=\bfseries,
title= Ejercicio~\thetcbcounter,
% CAJA LINEA MODIFICACIONES
boxed title style={
  sharp corners,
  rounded corners=northwest,
  rounded corners=northeast,
  % outer arc=0pt,
  % arc=0pt,
  },
% CONTENIDO MODIFICACIONES
colback = white,
fontupper = \itshape,
coltext =  black,
% MARCO MODIFICACIONES
rightrule=0mm,
toprule=0pt,
bottomrule= 0pt,
leftrule = 4pt
}

%%%%%%%%%%%%%%%%%%%%%%%%%%%%%%%%%%%%%%%%%%%%%%%%%%%%%%%%%%%%%%%%%%%%%%%%%%%%
%%%%%%%%%%%%%%%%%%%%%%%%%%%%%%%%%%%%%%%%%%%%%%%%%%%%%%%%%%%%%%%%%%%%%%%%%%%%
%%%%%%%%% CREE COMANDOS PARA FACILITAR ESCRITURA %%%%%%%%%%%%%%%%%%%%%%%%%%%
%%%%%%%%%%%%%%%%%%%%%%%%%%%%%%%%%%%%%%%%%%%%%%%%%%%%%%%%%%%%%%%%%%%%%%%%%%%%
%%%%%%%%%%%%%%%%%%%%%%%%%%%%%%%%%%%%%%%%%%%%%%%%%%%%%%%%%%%%%%%%%%%%%%%%%%%%

\newcommand{\I}[4]{\displaystyle\int\limits_#1^#2 #3 \,\text{d}#4}
\newcommand{\III}[2]{\displaystyle\int#1 \,\text{d}#2}
\newcommand{\II}[1]{\displaystyle\int#1 \,\text{d$x$}}
\newcommand{\fun}[3]{$#1:#2 \longrightarrow #3$}


%%%%%%%%%%%%%%%%%%%%%%%%%%%%%%%%%%%%%%%%%%%%%%%%%%%%%%%%%%%%%%%%%%%%%%%%%%%%
%%%%%%%%%%%%%%%%%%%%%%%%%%%%%%%%%%%%%%%%%%%%%%%%%%%%%%%%%%%%%%%%%%%%%%%%%%%%
%%%%%%%%%%%%%%%%% MODIFIQUE ALGUNOS COMANDOS %%%%%%%%%%%%%%%%%%%%%%%%%%%%%%%
%%%%%%%%%%%%%%%%%%%%% EL INTERLINEADO %%%%%%%%%%%%%%%%%%%%%%%%%%%%%%%%%%%%%%
%%%%%%%%%%%%%%%%%%%%%%%%%%%%%%%%%%%%%%%%%%%%%%%%%%%%%%%%%%%%%%%%%%%%%%%%%%%%
%%%%%%%%%%%%%%%%%%%%%%%%%%%%%%%%%%%%%%%%%%%%%%%%%%%%%%%%%%%%%%%%%%%%%%%%%%%%

\renewcommand{\baselinestretch}{1}

%%%%%%%%%%%%%%%%%%%%%%%%%%%%%%%%%%%%%%%%%%%%%%%%%%%%%%%%%%%%%%%%%%%%%%%%%%%%
%%%%%%%%%%%%%%%%%%%%%%%%%%%%%%%%%%%%%%%%%%%%%%%%%%%%%%%%%%%%%%%%%%%%%%%%%%%%
%%%%%%%%%%%%%%%%%% COLUMNAS ES AMBIENTE MULTICOLS %%%%%%%%%%%%%%%%%%%%%%%%%%
%%%%%%%%%%%%%%%%%%%%%%%%%%%%%%%%%%%%%%%%%%%%%%%%%%%%%%%%%%%%%%%%%%%%%%%%%%%%
%%%%%%%%%%%%%%%%%%%%%%%%%%%%%%%%%%%%%%%%%%%%%%%%%%%%%%%%%%%%%%%%%%%%%%%%%%%%

\setlength{\columnseprule}{1pt}
\def\columnseprulecolor{\color{darktangerine}}

%%%%%%%%%%%%%%%%%%%%%%%%%%%%%%%%%%%%%%%%%%%%%%%%%%%%%%%%%%%%%%%%%%%%%%%%%%%%
%%%%%%%%%%%%%%%%%%%%%%%%%%%%%%%%%%%%%%%%%%%%%%%%%%%%%%%%%%%%%%%%%%%%%%%%%%%%
%%%%% ESPACIO ENTRE RENGLONES,COLUMNAS MATRIX  Y THICK DE \FCOLORBOX %%%%%%%
%%%%%%%%%%%%%%%%%%%%%%%%%%%%%%%%%%%%%%%%%%%%%%%%%%%%%%%%%%%%%%%%%%%%%%%%%%%%
%%%%%%%%%%%%%%%%%%%%%%%%%%%%%%%%%%%%%%%%%%%%%%%%%%%%%%%%%%%%%%%%%%%%%%%%%%%%

\renewcommand{\arraystretch}{1.2} % for the vertical padding (space)
\setlength{\tabcolsep}{0.2 cm} % for the horizontal padding  (space)
\setlength{\fboxrule}{3pt}

%%%%%%%%%%%%%%%%%%%%%%%%%%%%%%%%%%%%%%%%%%%%%%%%%%%%%%%%%%%%%%%%%%%%%%%%%%%%
%%%%%%%%%%%%%%%%%%%%%%%%%%%%%%%%%%%%%%%%%%%%%%%%%%%%%%%%%%%%%%%%%%%%%%%%%%%%
%%%%%%%%%%%%%%%%%%%%%%%%%%% ESTILO DE LA PÁGINAS %%%%%%%%%%%%%%%%%%%%%%%%%%%
%%%%%%%%%%%%%%%%%%%%%%%%%%%%%%%%%%%%%%%%%%%%%%%%%%%%%%%%%%%%%%%%%%%%%%%%%%%%
%%%%%%%%%%%%%%%%%%%%%%%%%%%%%%%%%%%%%%%%%%%%%%%%%%%%%%%%%%%%%%%%%%%%%%%%%%%%

\pagestyle{fancy}
\fancyhf{}
\fancyhead[RE, RO]{}
\fancyhead[LE, LO]{}
\fancyfoot[CE,CO]{\thepage}
\fancyfoot[RE,RO]{\small{\textsc{Y. Sarahi García González}}}
\fancyfoot[LE,LO]{\small{\textsc{Reconocimiento estadístico de patrones}}}
\chead{\includegraphics[scale=.3]{/Users/ely/Documents/Plantilla/Figures/waves.pdf}}
\renewcommand{\headrulewidth}{0pt}
\renewcommand{\footrulewidth}{0pt}

%%%%%%%%%%%%%%%%%%%%%%%%%%%%%%%%%%%%%%%%%%%%%%%%%%%%%%%%%%%%%%%%%%%%%%%%%%%%
%%%%%%%%%%%%%%%%%%%%%%%%%%%%%%%%%%%%%%%%%%%%%%%%%%%%%%%%%%%%%%%%%%%%%%%%%%%%
%%%%%%%%%%%%% CAPÍTULOS MISMA PÁGINA %%%%%%%%%%%%%%%%%%%%%%%%%%%%%%%%%%%%%%%
%%%%%%%%%%%%%%%%%%%%%%%%%%%%%%%%%%%%%%%%%%%%%%%%%%%%%%%%%%%%%%%%%%%%%%%%%%%%
%%%%%%%%%%%%%%%%%%%%%%%%%%%%%%%%%%%%%%%%%%%%%%%%%%%%%%%%%%%%%%%%%%%%%%%%%%%%

\usepackage{etoolbox}
\makeatletter
\patchcmd{\chapter}{\if@openright\cleardoublepage\else\clearpage\fi}{}{}{}
\makeatother

%%%%%%%%%%%%%%%%%%%%%%%%%%%%%%%%%%%%%%%%%%%%%%%%%%%%%%%%%%%%%%%%%%%%%%%%%%%%
%%%%%%%%%%%%%%%%%%%%%%%%%%%%%%%%%%%%%%%%%%%%%%%%%%%%%%%%%%%%%%%%%%%%%%%%%%%%
%%%%%%%%%%%%%%%%%%%%%%%%%%%%% EMPEZAMOS %%%%%%%%%%%%%%%%%%%%%%%%%%%%%%%%%%%%
%%%%%%%%%%%%%%%%%%%%%%%%%%%%%%%%%%%%%%%%%%%%%%%%%%%%%%%%%%%%%%%%%%%%%%%%%%%%
%%%%%%%%%%%%%%%%%%%%%%%%%%%%%%%%%%%%%%%%%%%%%%%%%%%%%%%%%%%%%%%%%%%%%%%%%%%%

\begin{document}
\synctex=1 % PARA SINCRONIZAR PDF AL PRESIONAR
%%%%%%%%%%%%%%%%%%%%%%%%%%%%%%%%%%%%%%%%%%%%%%%%%%%%%%%%%%%%%%%%%%%%%%%%%%%%
% \begin{savequote}[45mm]
% ---Frase---
% \qauthor{Guillermo Gachuz Atitlán}
% \end{savequote}
%%%%%%%%%%%%%%%%%%%%%%%%%%%%%%%%%%%%%%%%%%%%%%%%%%%%%%%%%%%%%%%%%%%%%%%%%%%%
%%%%%%%%%%%%%%%%%%%%%%%%%%%%%%%%%%%%%%%%%%%%%%%%%%%%%%%%%%%%%%%%%%%%%%%%%%%%
\chapter*{\begin{tabular}{p{12cm}  c}
   \begin{flushright}
    Tarea IV\\\small{Y. Sarahi García González}
   \end{flushright} & \includegraphics[scale=0.3, raise =-2cm]{/Users/ely/Documents/Plantilla/Figures/cimat.png} \\
  \end{tabular} }
\vspace{-2cm}
%%%%%%%%%%%%%%%%%%%%%%%%%%%%%%%%%%%%%%%%%%%%%%%%%%%%%%%%%%%%%%%%%%%%%%%%%%%%
%%%%%%%%%%%%%%%%%%%%%%%%%%%%%%%%%%%%%%%%%%%%%%%%%%%%%%%%%%%%%%%%%%%%%%%%%%%


%%%%%%%%%%%%%%%%%%%%%%%%%%%%%%%%%%%%%%%%%%%%%%%%%%%%%%%%%%%%%%%%%%%%%%%%%%%%
%%%%%%%%%%%%%%%%%%%%%%%%%%%%%%%%%%%%%%%%%%%%%%%%%%%%%%%%%%%%%%%%%%%%%%%%%%%%
%%%%%%%%%%%%%%%%%%%%%%%%%%%%%%%%%%%%%%%%%%%%%%%%%%%%%%%%%%%%%%%%%%%%%%%%%%%%
%%%%%%%%%%%%%%%%%%%%%%%%%%%%%%%%%%%%%%%%%%%%%%%%%%%%%%%%%%%%%%%%%%%%%%%%%%%%
%%%%%%%%%%%%%%%%%%%%%%%%%%%%%%%%%%%%%%%%%%%%%%%%%%%%%%%%%%%%%%%%%%%%%%%%%%%%


{\color{mirosa}\section*{Problema 1}}
Supongamos que para un problema de clasificaci\'{o}n binaria, se construye un clasificador donde se permite, 
adem\'{a}s de regresar como predici\'{o}n $0$ y $1$,  tambi\'{e}n abstenerse.

El costo de predecir $1$ si la verdadera categoria es $0$, es 1 peso.   El costo de predecir $0$ si la verdadera categoria es $1$, tambi\'{e}n es 1 peso. El costo de abstenerse es $\theta$, una constante dada de antemano: $0 < \theta  < \frac{1}{2}$.

Calcula el clasificador Bayesiano \'{o}ptimo en funci\'{o}n de $\theta$ y $P(Y=1|X=x)$.   


{\color{mirosa}Sol.}

Definimos la función de costo, donde \textbf{\textit{abstenerse}} lo denotaremos con la etiqueta 2:

\begin{equation}
    L( Y=y,\hat{Y}= \hat{y}) =\begin{cases}0\quad \hat{y}=y, y\in\{0,1\} \\
        1 \quad \hat{y}\neq y, y\in\{0,1\}\\
        \theta \quad \hat{y}=2, y\in\{0,1\}\end{cases}
        \label{costo}
\end{equation}

Buscamos resolver:

\begin{equation*}
    \min _{\hat{Y}(x) }\left\{ E_{Y|X=x}L\left( Y,\hat{Y}(x) \right)\right\} 
\end{equation*}

Como nuestro caso es discreto, lo anterior es:

\begin{equation*}
    \min _{\hat{Y}(x) }\left\{ \sum_{y}L\left[Y,\hat{Y}(x)\right]  p\left(Y=y|X=x\right)\right\} 
\end{equation*}

A continuación, vamos a concentrarmos únicamente el la suma, a la que denotareos S.

\begin{equation*}
    S=L\left[Y=0,\hat{Y}(x)\right]  p\left(Y=0|X=x\right) + L\left[Y=1,\hat{Y}(x)\right]  p\left(Y=1|X=x\right)
\end{equation*}

Ahora, según tomemos $\hat{Y}$ \textbf{fijo}, el valor de esta suma cambia, 

\begin{itemize}
    \item $\hat{y}(x)=0\implies S=L[Y=1,\hat{Y}=0]*P(Y=1|X=x)$
    \item $\hat{y}(x)=1\implies S=L[Y=0,\hat{Y}=1]*P(Y=0|X=x)$
    \item $\hat{y}(x)=2\implies S=L[Y=0,\hat{Y}=2]*P(Y=0|X=x)+L[Y=1,\hat{Y}=2]*P(Y=1|X=x)$
\end{itemize}

Donde en cada caso tomando en cuenta de la función de costo (\ref{costo}) que $L\left[Y,\hat{Y}(x)\right]=0$ cuando $\hat{y}=y$.
Ahora, sustituyedo los demás valores de L:


\begin{itemize}
    \item $\hat{y}(x)=0\implies S=P(Y=1|X=x)$
    \item $\hat{y}(x)=1\implies S=P(Y=0|X=x)$
    \item $\hat{y}(x)=2\implies S=\theta*P(Y=0|X=x)+\theta*P(Y=1|X=x)$
\end{itemize}

Y usando la propiedad de probabilidad conjunta $P(A|B)=1-P(A^c|B)$


\begin{itemize}
    \item $\hat{y}(x)=0\implies S=P(Y=1|X=x)\equiv P$
    \item $\hat{y}(x)=1\implies S=1-P(Y=1|X=x)\equiv 1-P$
    \item $\hat{y}(x)=2\implies S=\theta$
\end{itemize}

Finalmente, veamos que las dos primeras posibilidades están directamente relacionadas, que sólo nos abstenemos si $\theta$ es menor
que el mínimo de $P$ y $P^c$, y que $P\in [0,1]$ y $\theta\in (0,\frac{1}{2})$, de manera que el  clasicador Bayesiano 
optimo en términos de $theta$ y de $P(Y=1|X=x)\equiv P$ es:


 \begin{equation}
   \hat{Y}(x) =
        \begin{cases}
        0 \quad P<0.5 , P<\theta\\
        1 \quad P>0.5 , P>\theta\\
        2 \quad \theta < min \{P,1-P\}
        \end{cases}
        \label{prediccion}
\end{equation}

%%%%%%%%%%%%%%%%%%%%%%%%%%%%%%%%%%%%%%%%%%%%%%%%%%%%%%%%%%%%%%%%%%%%%%%%%%%%
%%%%%%%%%%%%%%%%%%%%%%%%%%%%%%%%%%%%%%%%%%%%%%%%%%%%%%%%%%%%%%%%%%%%%%%%%%%%
%%%%%%%%%%%%%%%%%%%%%%%%%%%%%%%%%%%%%%%%%%%%%%%%%%%%%%%%%%%%%%%%%%%%%%%%%%%%
%%%%%%%%%%%%%%%%%%%%%%%%%%%%%%%%%%%%%%%%%%%%%%%%%%%%%%%%%%%%%%%%%%%%%%%%%%%%
%%%%%%%%%%%%%%%%%%%%%%%%%%%%%%%%%%%%%%%%%%%%%%%%%%%%%%%%%%%%%%%%%%%%%%%%%%%%

{\color{mirosa}\section*{Problema 2}}
Considera un problema de clasificaci\'{o}n binaria con predictores X.  Supongamos
que $P(Y = 1) = P(Y = 0)$ y que $P(X|Y = i)$ sigue una distribuci\'{o}n
Poisson con par\'{a}metro $\lambda_i$.

Derive el clasificador Bayesiano  \'{o}ptimo si el costo de un falso positivo
es dos veces el costo de un falso negativo.



{\color{mirosa}Sol.}

Si el costo de un falso positivo es dos veces el costo de un falso negativo, la matriz de confusión es:

\begin{tabular}{c|ccc}
    & $\hat{Y}=1$ & $\hat{Y}=0 $  \\ \hline
   Y=0 & 0 & $L_{FN}$ \\
   Y=1  & $2L_{FN}$ & 0 \\
\end{tabular}

Y el clasificdor Bayesiano en el caso binario está dado por: 

\begin{equation}
    \hat{Y}(x)=I\left[\frac{P(X=x|Y=1)}{P(X=x|Y=0)}>\frac{L(0,1)P(Y=0)}{L(1,0)P(Y=1)}\right]
\end{equation}
 

Pero sabemos que $$P(X=x|Y=i)=\frac{\lambda_{i}^x exp(\lambda_i)}{x!}$$

Además $P(Y=1)=P(Y=0)$ y $L(0,1)=2L(1,0)$, por lo que el clasificador Bayesiano óptimo binario es: 


\begin{equation}
    \hat{Y}=I\left[\frac{\lambda_{1}^x exp(\lambda_1)}{\lambda_{0}^x exp(\lambda_0)}>2\right]
\end{equation}


{\color{mirosa}\section*{Problema 3}}

Supongamos que $X,Y$ sean v.a. discretas:

\begin{tabular}{c|ccc}
 & X=0 & X=1 & X=2  \\ \hline
Y=0 & 0.1 & 0.3 & 0.25\\
Y=1  & 0.25 & 0.05 & 0.05\\
\end{tabular}

\begin{enumerate}
    \item Si $L(0,1)=L(1,0)$, calcula el clasificador Bayesiano \'optimo de $Y$ usando $X$.
    \item Si $L(0,1)=2L(1,0)$ calcula el clasificador Bayesiano \'optimo y su error (promedio) correspondiente.
\end{enumerate} 

{\color{mirosa}Sol.}

Buscamos:

\begin{equation}
    \hat{Y}(x)=I\left[\frac{P(X=x|Y=1)}{P(X=x|Y=0)}>\frac{L(0,1)P(Y=0)}{L(1,0)P(Y=1)}\right]
\end{equation}

Por lo que necesitamos las probabilidades mariginales y conjuntas de X y Y.
Primero, a partir de la tabla, obtenemos la distribución marginal de $X$, $P(X)=\sum_{y}P(Y=y,X=x)$, 


    $$
    P[X=x]=\left\{\begin{array}{cc}
    0.35 & \text { si } X=0 \\
    0.35 & \text { si } X=1 \\
    0.3 & \text { si } X=2
    \end{array}\right.
    $$

Y lo mismo para la distribución marginal de $Y$

    $$
    P[Y=y]=\left\{\begin{array}{cc}
    0.65 & \text { si }Y=0 \\
    0.35 & \text { si }Y=1 \\
    \end{array}\right.
    $$

Usando el Teorema de Bayes \ref{bayes} para la probabilidad condicional $Y \mid X$:
    
    \begin{equation}
        P(Y=y \mid X=x)=\frac{P(Y=y) P(X=x \mid Y=y)}{P(X=x)}=\frac{P(X=x, Y=x)}{P(X=x)}
        \label{bayes}
    \end{equation}
    

    \begin{tabular}{c| c c}
     & $Y=0$ & $Y=1$ \\
    \hline$Y \mid X=0$ & $\frac{2}{7}$ & $\frac{5}{7}$ \\
    \hline$Y \mid X=1$ & $\frac{6}{7}$ & $\frac{1}{7}$ \\
    \hline$Y \mid X=2$ & $\frac{5}{6}$ & $\frac{1}{6}$ \\
    \end{tabular}

    {\color{mirosa} $L(0,1)=L(1,0)$}
    
    El estimador está dado por:

    \begin{equation}
        \widehat{Y}(x)=I\left(\frac{P(Y=1 \mid X=x)}{P(Y=0 \mid X=x)}>\frac{12}{7}\right) 
    \end{equation}

    Usando la tabla de probabilidades conjuntas:
    
    $$
    \begin{gathered}
    \frac{P(Y=1 \mid X=0)}{P(Y=0 \mid X=0)}=\frac{5 / 7}{2 / 7}=\frac{5}{2}> \frac{12}{7}\\
    \frac{P(Y=1 \mid X=1)}{P(Y=0 \mid X=1)}=\frac{1 / 7}{6 / 7}=\frac{1}{6}< \frac{12}{7}\\
    \frac{P(Y=1 \mid X=2)}{P(Y=0 \mid X=2)}=\frac{1 / 6}{5 / 6}=\frac{1}{5}< \frac{12}{7}
    \end{gathered}
    $$

    Por lo que el clsificador Bayesiano optimo es:

    \begin{equation}
        \hat{Y}(x) =
             \begin{cases}
             1 \quad x=0\\
             0 \quad x\in\{1,2\}\\
             \end{cases}
     \end{equation}

    
     
    {\color{mirosa} $L(0,1)=2L(1,0)$}


    En este caso el estimador está dado por:

    \begin{equation}
        \widehat{Y}(x)=I\left(\frac{P(Y=1 \mid X=x)}{P(Y=0 \mid X=x)}>\frac{24}{7}\right) 
    \end{equation}

    Usando la tabla de probabilidades conjuntas:
    
    $$
    \begin{gathered}
    \frac{P(Y=1 \mid X=0)}{P(Y=0 \mid X=0)}=\frac{5 / 7}{2 / 7}=\frac{5}{2}<\frac{24}{7} \\
    \frac{P(Y=1 \mid X=1)}{P(Y=0 \mid X=1)}=\frac{1 / 7}{6 / 7}=\frac{1}{6}<\frac{24}{7} \\
    \frac{P(Y=1 \mid X=2)}{P(Y=0 \mid X=2)}=\frac{1 / 6}{5 / 6}=\frac{1}{5}<\frac{24}{7} 
    \end{gathered}
    $$

    Por lo que el clsificador Bayesiano optimo es:

    \begin{equation}
        \hat{Y}(x) = 0
     \end{equation}



    


    
     Ahora calculemos el error
     $$
     \begin{aligned}
     E[L(Y, \widehat{Y}(x))]& =  \sum_{y}L\left[Y,\hat{Y}(x)\right]  p\left(Y=y|X=x\right)\\
     & = L(0,1) P(Y=1 \mid X=1)P(X=1)+L(0,1) P(Y=1 \mid X=2)P(X=2)\\
     & =\frac{2}{7} 0.35 L(1,0)+\frac{1}{7} 0.35 L(0,1)+\frac{1}{6} 0.3 L(0,1) \\
     & =L(1,0) \frac{1}{10}+L(0,1) \frac{1}{10}=0.3 L(1,0)
     \end{aligned}
     $$




%%%%%%%%%%%%%%%%%%%%%%%%%%%%%%%%%%%%%%%%%%%%%%%%%%%%%%%%%%%%%%%%%%%%%%%%%%%%
%%%%%%%%%%%%%%%%%%%%%%%%%%%%%%%%%%%%%%%%%%%%%%%%%%%%%%%%%%%%%%%%%%%%%%%%%%%%
%%%%%%%%%%%%%%%%%%%%%%%%%%%%%%%%%%%%%%%%%%%%%%%%%%%%%%%%%%%%%%%%%%%%%%%%%%%%
%%%%%%%%%%%%%%%%%%%%%%%%%%%%%%%%%%%%%%%%%%%%%%%%%%%%%%%%%%%%%%%%%%%%%%%%%%%%
%%%%%%%%%%%%%%%%%%%%%%%%%%%%%%%%%%%%%%%%%%%%%%%%%%%%%%%%%%%%%%%%%%%%%%%%%%%%
\end{document}
