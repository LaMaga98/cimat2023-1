\documentclass[12pt]{book}   
\usepackage{graphics}
\usepackage{spaccent}  
\usepackage{amsfonts}
\usepackage[spanish]{babel} 
\usepackage{fancybox, calc}
\newcommand {\?}{?`}  
\newcommand{\B}{\mathbb{B}}
\newcommand{\M}{\mathbb{M}}
\newcommand{\J}{\mathbb{J}}
\newcommand{\K}{\mathbb{K}}
\newcommand{\X}{\mathbb{X}}
\begin{document}
\specialaccent \mbox{} 
\begin{center}
{\Large \bf Tarea 4 (parte 1)} \mbox{} \vspace{0.5cm} \\
Entregar a m\'as tardar el martes 8PM
\end{center}


\begin{enumerate}
\item 
Supongamos que para un problema de clasificaci\'{o}n binaria, se construye un clasificador donde se permite, 
adem\'{a}s de regresar como predici\'{o}n $0$ y $1$,  tambi\'{e}n abstenerse.

El costo de predecir $1$ si la verdadera categoria es $0$, es 1 peso.   El costo de predecir $0$ si la verdadera categoria es $1$, tambi\'{e}n es 1 peso. El costo de abstenerse es $\theta$, una constante dada de antemano: $0 < \theta  < \frac{1}{2}$.

Calcula el clasificador Bayesiano \'{o}ptimo en funci\'{o}n de $\theta$ y $P(Y=1|X=x)$.   

\item
Considera un problema de clasificaci\'{o}n binaria con predictores X.  Supongamos
que $P(Y = 1) = P(Y = 0)$ y que $P(X|Y = i)$ sigue una distribuci\'{o}n
Poisson con par\'{a}metro $\lambda_i$.

Derive el clasificador Bayesiano  \'{o}ptimo si el costo de un falso positivo
es dos veces el costo de un falso negativo.


\item
Supongamos que $X,Y$ sean v.a. discretas:

\begin{tabular}{c|ccc}
 & X=0 & X=1 & X=2  \\ \hline
Y=0 & 0.1 & 0.3 & 0.25\\
Y=1  & 0.25 & 0.05 & 0.05\\
\end{tabular}

Si $L(0,1)=L(1,0)$, calcula el clasificador Bayesiano \'optimo de $Y$ usando $X$.\\
Si $L(0,1)=2L(1,0)$ calcula el clasificador Bayesiano \'optimo y su error (promedio) correspondiente. \\


\item (no entregar)
Calcula el clasificador Bayesiano \'optimo para una funci\'on de costo sim\'etrico, $Y|X=x \sim {\cal N}(\mu, \sigma^2_y), y\in \{0,1\}$ y $P(Y=1)=P(Y=0)$.


\item  (no entregar)
Para un problema de clasificaci\'on binaria y $x\in {\cal R}^2$, dibuja un conjunto de datos con tres observaciones donde el clasificador 1-NN tiene un error emp\'{i}rica (sobre el conjunto de entrenamiento) que no sea cero.


\end{enumerate}

\end{document}
