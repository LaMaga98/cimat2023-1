%%%%%%%%%%%%%%%%%%%%%%%%%%%%%%%%%%%%%%%%%%%%%%%%%%%%%%%%%%%%%%%%%%%%%%%%%%%%
%%%%%%%%%%%%%%%%%%%%%%%%%%%%%%%%%%%%%%%%%%%%%%%%%%%%%%%%%%%%%%%%%%%%%%%%%%%%
%%%%%%%%%%%%%%%%%%%%%%%%% PAQUETES QUE UTILIZO %%%%%%%%%%%%%%%%%%%%%%%%%%%%%
%%%%%%%%%%%%%%%%%%%%%%%%%%%%%%%%%%%%%%%%%%%%%%%%%%%%%%%%%%%%%%%%%%%%%%%%%%%%
%%%%%%%%%%%%%%%%%%%%%%%%%%%%%%%%%%%%%%%%%%%%%%%%%%%%%%%%%%%%%%%%%%%%%%%%%%%%

\documentclass[letter, 11pt, twoside]{report}
\usepackage{amsthm}
\usepackage[many]{tcolorbox}
\usepackage{thmtools}
\usepackage{amssymb,bm,amsfonts,amsmath}
\usepackage[utf8]{inputenc}
\usepackage[spanish]{babel}
\usepackage[export]{adjustbox}
\usepackage{hyperref}
\usepackage{enumerate}
\usepackage{makeidx}
\usepackage{float}
\usepackage{graphicx, import}
\usepackage{subfig}
\usepackage{upgreek}
\usepackage{float}
\usepackage[all]{xy}
\usepackage{thmtools}
\usepackage{titlesec}
\usepackage{mathrsfs}
\usepackage{multicol}
\usepackage{tikz-cd}
\usetikzlibrary{patterns}
\usetikzlibrary{plotmarks}
\usepackage{wrapfig}
\usepackage{stmaryrd}
\usepackage{subfloat}
\usepackage{svg}
\usepackage{yfonts}
\usepackage{fancyhdr}
\usepackage{pifont}
\usepackage{pdfpages}
\usepackage{ marvosym }
\usepackage{hyperref}
\usepackage{pdflscape}
\usepackage{setspace}
\usepackage{color}
\usepackage{bm}
\usepackage{epigraph}
\usepackage{quotchap}
\usepackage[framemethod=TikZ]{mdframed}
\usepackage[nottoc,numbib]{tocbibind}
\usepackage[customcolors]{hf-tikz}
\usetikzlibrary{babel}
% PARA VER LAS REFERENCIAS LABELS
% \usepackage[notcite,color]{showkeys}
% CHECA http://www.tug.dk/FontCatalogue/iwonalightcondensed/
\usepackage[light,math]{iwona}
\usepackage[T1]{fontenc}
\usepackage{MnSymbol}
\usepackage{varwidth} % CAJA DE EJERCICIOS Y \SOMBREADO
\tcbuselibrary{vignette,many}
\tcbuselibrary{skins}
\usepackage{pgfplots}
\usepgfplotslibrary{fillbetween}
\pgfplotsset{compat=1.16}
\usepackage{xcolor}
% PARA ESCRIBIS CÓDIGO Y PSEUDOCÓDIGO
\usepackage{algorithm}
\usepackage{algpseudocode}
\usepackage{listings}
\usepackage{color, xcolor}


%%%%%%%%%%%%%%%%%%%%%%%%%%%%%%%%%%%%%%%%%%%%%%%%%%%%%%%%%%%%%%%%%%%%%%%%%%%%
%%%%%%%%%%%%%%%%%%%%%%%%%%%%%%%%%%%%%%%%%%%%%%%%%%%%%%%%%%%%%%%%%%%%%%%%%%%%
%%%%%%%%%%%%%%%%%%%%%%%%% MEMO PYTHON Y C %%%%%%%%%%%%%%%%%%%%%%%%%%%%%%%%%%
%%%%%%%%%%%%%%%%%%%%%%%%%%%%%%%%%%%%%%%%%%%%%%%%%%%%%%%%%%%%%%%%%%%%%%%%%%%%
%%%%%%%%%%%%%%%%%%%%%%%%%%%%%%%%%%%%%%%%%%%%%%%%%%%%%%%%%%%%%%%%%%%%%%%%%%%%

%CÓMO QUEDERÁ EL COLOREADO Y HIGHLIGHT DEL CÓDIGO
\definecolor{dkgreen}{rgb}{0.9,0.6,0.8}
\definecolor{blue}{rgb}{0.0,0.49,0.4}
\definecolor{gray97}{gray}{.97}
\definecolor{gray75}{gray}{.75}
\definecolor{gray45}{gray}{.45}
\definecolor{codepurple}{rgb}{0.58,0,0.82}
\definecolor{backcolour}{rgb}{0.95,0.95,0.92}
\definecolor{codegreen}{rgb}{0,0.6,0}
\definecolor{codegray}{rgb}{0.5,0.5,0.5}

\lstdefinestyle{mystyle}{
    backgroundcolor=\color{gray97},
    commentstyle=\color{cyan!75!black},
    keywordstyle=\color{magenta},
    numberstyle=\tiny\color{codegray},
    stringstyle=\color{codepurple},
    basicstyle=\ttfamily\footnotesize,
    breakatwhitespace=false,
    breaklines= true,
    captionpos=b,
    keepspaces=true,
    numbers=left,
    numbersep=5pt,
    showspaces=false,
    showstringspaces=false,
    showtabs=false,
    tabsize=2,
    language=bash,   %% PHP, C, Java, etc... bash is the standard
    extendedchars=true,
    inputencoding=latin1
}

\lstset{style=mystyle, literate =
                        {í}{{\'i}}1
                        {á}{{\'a}}1
                        {é}{{\'e}}1
                        {ó}{{\'o}}1
                        {ú}{{\'u}}1
                        {ñ}{{\~n}}1
                        {ü}{{\"u}}1
                            }

%%%%%%%%%%%%%%%%%%%%%%%%%%%%%%%%%%%%%%%%%%%%%%%%%%%%%%%%%%%%%%%%%%%%%%%%%%%%
%%%%%%%%%%%%%%%%%%%%%%%%%%%%%%%%%%%%%%%%%%%%%%%%%%%%%%%%%%%%%%%%%%%%%%%%%%%%
%%%%%%%%%%%%%%% COLORES DEL PAQUETE SHOWKEYS %%%%%%%%%%%%%%%%%%%%%%%%%%%%%%%
%%%%%%%%%%%%%%%%%%%%%%%%%%%%%%%%%%%%%%%%%%%%%%%%%%%%%%%%%%%%%%%%%%%%%%%%%%%%
%%%%%%%%%%%%%%%%%%%%%%%%%%%%%%%%%%%%%%%%%%%%%%%%%%%%%%%%%%%%%%%%%%%%%%%%%%%%

\definecolor{refkey}{rgb}{255,0,0}
\definecolor{labelkey}{rgb}{255,0,0}
\definecolor{mirosa}{HTML}{FF007F}

%%%%%%%%%%%%%%%%%%%%%%%%%%%%%%%%%%%%%%%%%%%%%%%%%%%%%%%%%%%%%%%%%%%%%%%%%%%%
%%%%%%%%%%%%%%%%%%%%%%%%%%%%%%%%%%%%%%%%%%%%%%%%%%%%%%%%%%%%%%%%%%%%%%%%%%%%
%%%%%%%%%%%%%%%% MARGENES, VIENE EN EL MANUAL DE LATEX %%%%%%%%%%%%%%%%%%%%%
%%%%%%%%%%%%%%%% FORMATO ME LO PASO RO %%%%%%%%%%%%%%%%%%%%%%%%%%%%%%%%%%%%%
%%%%%%%%%%%%%%%%%%%%%%%%%%%%%%%%%%%%%%%%%%%%%%%%%%%%%%%%%%%%%%%%%%%%%%%%%%%%
%%%%%%%%%%%%%%%%%%%%%%%%%%%%%%%%%%%%%%%%%%%%%%%%%%%%%%%%%%%%%%%%%%%%%%%%%%%%

\parskip=5pt
\hoffset = 0pt
\headsep = 1.5 cm % estaba en 1.5 cm, lo cambie para el header de la imagen
\oddsidemargin = .5cm
\evensidemargin = .5cm
\textheight = 657pt
\textwidth = 15.6cm
\topmargin = -2 cm
\parindent=0mm

%%%%%%%%%%%%%%%%%%%%%%%%%%%%%%%%%%%%%%%%%%%%%%%%%%%%%%%%%%%%%%%%%%%%%%%%%%%%
%%%%%%%%%%%%%%%%%%%%%%%%%%%%%%%%%%%%%%%%%%%%%%%%%%%%%%%%%%%%%%%%%%%%%%%%%%%%
%%%%%%%%%%%%%%%%%%%%%%%% CREACIÓN DE EJERCICIO %%%%%%%%%%%%%%%%%%%%%%%%%%%%%
%%%%%%%%%%%%%%%%%% MODIFICACIÓN PROOF Y QED %%%%%%%%%%%%%%%%%%%%%%%%%%%%%%%%
%%%%%%%%%%%%%%%%%%%%%%%%%%%%%%%%%%%%%%%%%%%%%%%%%%%%%%%%%%%%%%%%%%%%%%%%%%%%
%%%%%%%%%%%%%%%%%%%%%%%%%%%%%%%%%%%%%%%%%%%%%%%%%%%%%%%%%%%%%%%%%%%%%%%%%%%%

\renewcommand{\qedsymbol}{\tiny{$\blacksquare$}}

\newenvironment{solucion}{\begin{proof}[\textcolor{magenta}{Solución}]}{\end{proof}}

\newtcolorbox[auto counter]{ejercicio}[1][]{
% ESTO ES PARA LA CAJA GENERAL
breakable, % por si cambias de pagina
enhanced, % estilo general
% TITULO MODIFICACIONES
coltitle= black,
colbacktitle= white,
titlerule= 0mm,
colframe = magenta,
fonttitle=\bfseries,
title= Ejercicio~\thetcbcounter,
% CAJA LINEA MODIFICACIONES
boxed title style={
  sharp corners,
  rounded corners=northwest,
  rounded corners=northeast,
  % outer arc=0pt,
  % arc=0pt,
  },
% CONTENIDO MODIFICACIONES
colback = white,
fontupper = \itshape,
coltext =  black,
% MARCO MODIFICACIONES
rightrule=0mm,
toprule=0pt,
bottomrule= 0pt,
leftrule = 4pt
}

%%%%%%%%%%%%%%%%%%%%%%%%%%%%%%%%%%%%%%%%%%%%%%%%%%%%%%%%%%%%%%%%%%%%%%%%%%%%
%%%%%%%%%%%%%%%%%%%%%%%%%%%%%%%%%%%%%%%%%%%%%%%%%%%%%%%%%%%%%%%%%%%%%%%%%%%%
%%%%%%%%% CREE COMANDOS PARA FACILITAR ESCRITURA %%%%%%%%%%%%%%%%%%%%%%%%%%%
%%%%%%%%%%%%%%%%%%%%%%%%%%%%%%%%%%%%%%%%%%%%%%%%%%%%%%%%%%%%%%%%%%%%%%%%%%%%
%%%%%%%%%%%%%%%%%%%%%%%%%%%%%%%%%%%%%%%%%%%%%%%%%%%%%%%%%%%%%%%%%%%%%%%%%%%%

\newcommand{\I}[4]{\displaystyle\int\limits_#1^#2 #3 \,\text{d}#4}
\newcommand{\III}[2]{\displaystyle\int#1 \,\text{d}#2}
\newcommand{\II}[1]{\displaystyle\int#1 \,\text{d$x$}}
\newcommand{\fun}[3]{$#1:#2 \longrightarrow #3$}


%%%%%%%%%%%%%%%%%%%%%%%%%%%%%%%%%%%%%%%%%%%%%%%%%%%%%%%%%%%%%%%%%%%%%%%%%%%%
%%%%%%%%%%%%%%%%%%%%%%%%%%%%%%%%%%%%%%%%%%%%%%%%%%%%%%%%%%%%%%%%%%%%%%%%%%%%
%%%%%%%%%%%%%%%%% MODIFIQUE ALGUNOS COMANDOS %%%%%%%%%%%%%%%%%%%%%%%%%%%%%%%
%%%%%%%%%%%%%%%%%%%%% EL INTERLINEADO %%%%%%%%%%%%%%%%%%%%%%%%%%%%%%%%%%%%%%
%%%%%%%%%%%%%%%%%%%%%%%%%%%%%%%%%%%%%%%%%%%%%%%%%%%%%%%%%%%%%%%%%%%%%%%%%%%%
%%%%%%%%%%%%%%%%%%%%%%%%%%%%%%%%%%%%%%%%%%%%%%%%%%%%%%%%%%%%%%%%%%%%%%%%%%%%

\renewcommand{\baselinestretch}{1}

%%%%%%%%%%%%%%%%%%%%%%%%%%%%%%%%%%%%%%%%%%%%%%%%%%%%%%%%%%%%%%%%%%%%%%%%%%%%
%%%%%%%%%%%%%%%%%%%%%%%%%%%%%%%%%%%%%%%%%%%%%%%%%%%%%%%%%%%%%%%%%%%%%%%%%%%%
%%%%%%%%%%%%%%%%%% COLUMNAS ES AMBIENTE MULTICOLS %%%%%%%%%%%%%%%%%%%%%%%%%%
%%%%%%%%%%%%%%%%%%%%%%%%%%%%%%%%%%%%%%%%%%%%%%%%%%%%%%%%%%%%%%%%%%%%%%%%%%%%
%%%%%%%%%%%%%%%%%%%%%%%%%%%%%%%%%%%%%%%%%%%%%%%%%%%%%%%%%%%%%%%%%%%%%%%%%%%%

\setlength{\columnseprule}{1pt}
\def\columnseprulecolor{\color{darktangerine}}

%%%%%%%%%%%%%%%%%%%%%%%%%%%%%%%%%%%%%%%%%%%%%%%%%%%%%%%%%%%%%%%%%%%%%%%%%%%%
%%%%%%%%%%%%%%%%%%%%%%%%%%%%%%%%%%%%%%%%%%%%%%%%%%%%%%%%%%%%%%%%%%%%%%%%%%%%
%%%%% ESPACIO ENTRE RENGLONES,COLUMNAS MATRIX  Y THICK DE \FCOLORBOX %%%%%%%
%%%%%%%%%%%%%%%%%%%%%%%%%%%%%%%%%%%%%%%%%%%%%%%%%%%%%%%%%%%%%%%%%%%%%%%%%%%%
%%%%%%%%%%%%%%%%%%%%%%%%%%%%%%%%%%%%%%%%%%%%%%%%%%%%%%%%%%%%%%%%%%%%%%%%%%%%

\renewcommand{\arraystretch}{1.2} % for the vertical padding (space)
\setlength{\tabcolsep}{0.2 cm} % for the horizontal padding  (space)
\setlength{\fboxrule}{3pt}

%%%%%%%%%%%%%%%%%%%%%%%%%%%%%%%%%%%%%%%%%%%%%%%%%%%%%%%%%%%%%%%%%%%%%%%%%%%%
%%%%%%%%%%%%%%%%%%%%%%%%%%%%%%%%%%%%%%%%%%%%%%%%%%%%%%%%%%%%%%%%%%%%%%%%%%%%
%%%%%%%%%%%%%%%%%%%%%%%%%%% ESTILO DE LA PÁGINAS %%%%%%%%%%%%%%%%%%%%%%%%%%%
%%%%%%%%%%%%%%%%%%%%%%%%%%%%%%%%%%%%%%%%%%%%%%%%%%%%%%%%%%%%%%%%%%%%%%%%%%%%
%%%%%%%%%%%%%%%%%%%%%%%%%%%%%%%%%%%%%%%%%%%%%%%%%%%%%%%%%%%%%%%%%%%%%%%%%%%%

\pagestyle{fancy}
\fancyhf{}
\fancyhead[RE, RO]{}
\fancyhead[LE, LO]{}
\fancyfoot[CE,CO]{\thepage}
\fancyfoot[RE,RO]{\small{\textsc{Y. Sarahi García González}}}
\fancyfoot[LE,LO]{\small{\textsc{Procesamiento de Lenguaje Natural}}}
\chead{\includegraphics[scale=.3]{/Users/ely/Documents/Plantilla/Figures/waves.pdf}}
\renewcommand{\headrulewidth}{0pt}
\renewcommand{\footrulewidth}{0pt}

%%%%%%%%%%%%%%%%%%%%%%%%%%%%%%%%%%%%%%%%%%%%%%%%%%%%%%%%%%%%%%%%%%%%%%%%%%%%
%%%%%%%%%%%%%%%%%%%%%%%%%%%%%%%%%%%%%%%%%%%%%%%%%%%%%%%%%%%%%%%%%%%%%%%%%%%%
%%%%%%%%%%%%% CAPÍTULOS MISMA PÁGINA %%%%%%%%%%%%%%%%%%%%%%%%%%%%%%%%%%%%%%%
%%%%%%%%%%%%%%%%%%%%%%%%%%%%%%%%%%%%%%%%%%%%%%%%%%%%%%%%%%%%%%%%%%%%%%%%%%%%
%%%%%%%%%%%%%%%%%%%%%%%%%%%%%%%%%%%%%%%%%%%%%%%%%%%%%%%%%%%%%%%%%%%%%%%%%%%%

\usepackage{etoolbox}
\makeatletter
\patchcmd{\chapter}{\if@openright\cleardoublepage\else\clearpage\fi}{}{}{}
\makeatother

%%%%%%%%%%%%%%%%%%%%%%%%%%%%%%%%%%%%%%%%%%%%%%%%%%%%%%%%%%%%%%%%%%%%%%%%%%%%
%%%%%%%%%%%%%%%%%%%%%%%%%%%%%%%%%%%%%%%%%%%%%%%%%%%%%%%%%%%%%%%%%%%%%%%%%%%%
%%%%%%%%%%%%%%%%%%%%%%%%%%%%% EMPEZAMOS %%%%%%%%%%%%%%%%%%%%%%%%%%%%%%%%%%%%
%%%%%%%%%%%%%%%%%%%%%%%%%%%%%%%%%%%%%%%%%%%%%%%%%%%%%%%%%%%%%%%%%%%%%%%%%%%%
%%%%%%%%%%%%%%%%%%%%%%%%%%%%%%%%%%%%%%%%%%%%%%%%%%%%%%%%%%%%%%%%%%%%%%%%%%%%

\begin{document}
\synctex=1 % PARA SINCRONIZAR PDF AL PRESIONAR
%%%%%%%%%%%%%%%%%%%%%%%%%%%%%%%%%%%%%%%%%%%%%%%%%%%%%%%%%%%%%%%%%%%%%%%%%%%%
% \begin{savequote}[45mm]
% ---Frase---
% \qauthor{Guillermo Gachuz Atitlán}
% \end{savequote}
%%%%%%%%%%%%%%%%%%%%%%%%%%%%%%%%%%%%%%%%%%%%%%%%%%%%%%%%%%%%%%%%%%%%%%%%%%%%
%%%%%%%%%%%%%%%%%%%%%%%%%%%%%%%%%%%%%%%%%%%%%%%%%%%%%%%%%%%%%%%%%%%%%%%%%%%%
\chapter*{\begin{tabular}{p{12cm}  c}
   \begin{flushright}
    Segundo Examen Parcial\\\small{Y. Sarahi Grcía González}
   \end{flushright} & \includegraphics[scale=0.3, raise =-2cm]{/Users/ely/Documents/Plantilla/Figures/cimat.png} \\
  \end{tabular} }
\vspace{-2cm}
%%%%%%%%%%%%%%%%%%%%%%%%%%%%%%%%%%%%%%%%%%%%%%%%%%%%%%%%%%%%%%%%%%%%%%%%%%%%
%%%%%%%%%%%%%%%%%%%%%%%%%%%%%%%%%%%%%%%%%%%%%%%%%%%%%%%%%%%%%%%%%%%%%%%%%%%


%%%%%%%%%%%%%%%%%%%%%%%%%%%%%%%%%%%%%%%%%%%%%%%%%%%%%%%%%%%%%%%%%%%%%%%%%%%%
%%%%%%%%%%%%%%%%%%%%%%%%%%%%%%%%%%%%%%%%%%%%%%%%%%%%%%%%%%%%%%%%%%%%%%%%%%%%
%%%%%%%%%%%%%%%%%%%%%%%%%%%%%%%%%%%%%%%%%%%%%%%%%%%%%%%%%%%%%%%%%%%%%%%%%%%%
%%%%%%%%%%%%%%%%%%%%%%%%%%%%%%%%%%%%%%%%%%%%%%%%%%%%%%%%%%%%%%%%%%%%%%%%%%%%
%%%%%%%%%%%%%%%%%%%%%%%%%%%%%%%%%%%%%%%%%%%%%%%%%%%%%%%%%%%%%%%%%%%%%%%%%%%%


\begin{enumerate}
    \item \textbf{¿De los sitios turísticos, cuál diría usted que es el más polémico y \textit{la razón de ello}?}
    Considero que el sitio más polémico es el Museo de las Momias. Esto puede verse en el problema 1, es el lugar con opiniones más variadas cómo se observa en los gráficos de frecuencia.
    Hay varios factores que polarizan las opiniones. Por un lado, muchos visitantes consideran el museo educativo, destacando su singularidad como puede verse en las nubes de palabras. Sin embargo, un segmento significativo de los visitantes tiene una opinión negativa, encontrando el museo algo inquietante. Algunos de los visitantes creen que la exhibición es irrespetuosa y explota los cuerpos de manera inapropiada, lo que genera fuertes reacciones negativas. También hay quienes dicen que es una estafa.
    \item \textbf{En cuanto al sitio más polémico, ¿Cómo es la diferencia de opinión y temas entre turistas nacionales e internacionales?}
    Los turistas nacionales suelen valorar más la conexión histórica y cultural del museo con la región. Esto se aprecia en las palabras más destacadas del ejercicio 3 pues hay palabras como "familia", "descubrimiento", "guanajuato", "ciudad". En contraste, los turistas internacionales, pueden encontrar el museo más perturbador o incluso ofensivo como se observa en algunas de las opiniones negativas de este lugar (su mayoría en inglés). Los temas de sus opiniones negativas incluyen críticas sobre la ética de exhibir cuerpos según ellos NO momificados y cuestionamientos sobre la sensibilidad cultural.
    
    \item \textbf{¿Cuál diría que es el sitio que le gusta más a las mujeres y por qué?}
    Yo diría que el sitio que más gusta a las mujeres es el Jardín de la Unión. En general este lugar es muy bien calificado (de acuerdo con lo que se refleja de las opiniones en las nubes de palabras) por su ambiente tranquilo y estéticamente agradable. Las opiniones generales destacan la belleza, el diseño y la atmósfera relajante.
    Entre las palabras más importantes obtenidas en el ejercicio 3 para éste lugar se destacan: safe, walking , pottery, cafeteria, bistros e iglesias. Revisando algunas de las opiniones de mujeres en se este lugar, se destaca que variedad de cafeterías, iglesias y restaurantes alrededor del jardín añade un atractivo adicional. Lo que va muy de acuerdo con las palabras relevantes de los tópicos encontrados con LSA.

    \item \textbf{¿Cuál diría que es el sitio que le gusta más a las personas jóvenes y por qué?}
    
    Yo diría que el Callejón del Beso es el sitio preferido por las personas jóvenes, para empezar, en el ejercicio 1 en las grpaficas de densidad por lugar podemos notar que este es el lugar con más personas menores de 25 años.
    Lo que tiene sentido pues el lugar tiene una fuerte atracción debido a la historia romántica y la leyenda del beso. 

    Las palabras que más destacan en los jóvenes para el callejón del beso son: story, beso,gift, balcony,fotografía,leyenda,dos,pareja, love, luck y enamorados.
    Todas ellas con una connotacón positiva y referente al amor de pareja. 
    \item \textbf{¿Qué otras observaciones valiosas puede obtener de su análisis? (e.g., ¿identificó de qué se queja la gente? ¿qué tipo de cosas le gustó a la gente?, etc.)}
    
    Una observación que me llamó la ayención es que Alhóndiga de Granaditas recibe opiniones diversas, por una lado algunas positivas,relacionadas con su importancia histórica y la calidad de las exposiciones. Lo que más se elogía es el valor educativo e histórico así como el mantenimiento del edificio. Por otro lado se menciona la falta de guías informativas en varios idiomas y la falta de limpieza en los alrededores.
    
    Otra observación que me interesó es que el Teatro Juárez es frecuentemente elogiado por su arquitectura y las obras, musicales y conciertos, siendo el lugar mejor calificado y el favorito. Las quejas comunes incluyen problemas de accesibilidad y la falta de señalización adecuada en inglés, lo cual tiene sentido pues como se observa en los histogramas de visitantes nacionales e internacionales, en todos los sitios hay casi tantos visitantes internacionales como nacionales. La falta de señalización en inglés puede afectar la experiencia. 
    
    En general, los visitantes valoran la autenticidad y preservación de los sitios, pero resaltan la necesidad de mejorar servicios y accesibilidad.
\end{enumerate}






%%%%%%%%%%%%%%%%%%%%%%%%%%%%%%%%%%%%%%%%%%%%%%%%%%%%%%%%%%%%%%%%%%%%%%%%%%%%
%%%%%%%%%%%%%%%%%%%%%%%%%%%%%%%%%%%%%%%%%%%%%%%%%%%%%%%%%%%%%%%%%%%%%%%%%%%%
%%%%%%%%%%%%%%%%%%%%%%%%%%%%%%%%%%%%%%%%%%%%%%%%%%%%%%%%%%%%%%%%%%%%%%%%%%%%
%%%%%%%%%%%%%%%%%%%%%%%%%%%%%%%%%%%%%%%%%%%%%%%%%%%%%%%%%%%%%%%%%%%%%%%%%%%%
%%%%%%%%%%%%%%%%%%%%%%%%%%%%%%%%%%%%%%%%%%%%%%%%%%%%%%%%%%%%%%%%%%%%%%%%%%%%
\end{document}
