\documentclass{article}
\usepackage{graphicx} % Required for inserting images
\usepackage{amsmath}
\usepackage{amsfonts}
\usepackage{hyperref}
\newcommand{\R}{\mathbb{R}}
\newcommand{\define}{\overset{def}{=}}


\title{Tarea 1. Métodos de Descenso para Funciones de Base Radial}
\author{Mariano Rivera}
\date{Fecha de entrega: 28 February 2024}

\begin{document}

\maketitle

Considere la imagen bidimensional $f$ de dimensioned${N \times N}$, que toma valores en el internamo $[0,1]$ y si $x= [x_1,x_2]^\top$ representas las coordenadas de cada pixel, entonces podemos aproximar la imagen como una suma de funciones base . Esto es:
\begin{equation}
f(x) = \sum_{j=1}^J \alpha_j \phi(x; \theta_j) + \eta(x)    
\label{eq:f}
\end{equation}
donde el vector $\alpha = [\alpha_1, \alpha_2, \ldots, \alpha_J ]$ son los coeficientes que pesan la contribución de cada función base, que se distiguen entre ellas por sus parámetros individuales $\theta = [\theta_1, \theta_2, \ldots, \theta_J ]$, y $\eta$ es un residual. Luego definimos $\phi_j \define \phi(x, \theta_j)$ y
\begin{equation}
    \label{eq_phi}
    \Phi_\theta \define \begin{bmatrix}
        \phi_1 \\
        \phi_2 \\
        \vdots \\
        \phi_J
    \end{bmatrix}
\end{equation}
Luego ecuación \eqref{eq:f} la podemos escribir en forma matricial como
$$
f = \Phi_{\theta} \alpha + \eta
$$
Para ajustar los parámetros y coeficientes de las funciones base resolvemos el problema no lineal:
\begin{equation}
    \underset{\theta, \alpha}{\arg\min} \| f - \Phi_{\theta} \alpha \|_2^2
\end{equation}
Esta optimización se realiza en dos pasos para lo que se requiere unos parámetros iniciales $\theta$:

{\bf Paso I}. Paso lineal, asumiendo ficjos los parámetros $\theta$, resolver el problema de mínimos cuadrados lineales (por ejemplo, usando la psuedoinversa de Moore-Penrose):
\begin{equation}
    \underset{\alpha}{\arg\min} \| f - \Phi_{\theta} \alpha \|_2^2
\end{equation}
para los coeficientes.

{\bf Paso II}. Paso no-lineal, asumiendo fijos los coeficientes $\alpha$, dar una actualización de descenso de gradiente en el problema de mínimos cuadrados no-lineales:
\begin{equation}
    \underset{\theta}{\arg\min} \| f - \Phi_{\theta} \alpha \|_2^2
\end{equation}
para los parámetros $\theta$.

Entre las funciones radiales mas populares están la multiquádrica:
\begin{equation}
    \label{eq:M1}
    \phi_j = \sqrt{r^2  + \kappa }
\end{equation}
y la Gaussiana
\begin{equation}
    \label{eq:MQ}
    \phi_j = \exp{(-\kappa r^2)}
\end{equation}
donde $r = (\theta_{j} - x).$ Note que los parámetros de la función radial son las coordenadas $\theta_j$; centro de la función radial. $\kappa$ es un parámetro de escala que se da.

Resolver el problema de ajuste de RBFs para:

\begin{itemize}
    \item Usar una imagen momocromática (tonos de gris) de 256x256 pixeles.

    \item Usar entre 100 a 500 funciones radiales. Encuentre el compromiso que le parezca adecuado entre buena reconstrucción y rapidez en la reconstrucción, esto es a su criterio.

    \item Los centros $\theta$ de las funciones radiales se inicializan aleatoriamente en el intervalo [1,N]


    \item Busque un valor de $\kappa$ adecuado para la imagen de prueba que seleccione.
     
    \item Ajustar el modelo usando la multiquádrica y compara con la la Gaussiana

    \item Usar los métodos de descenso de gradiente: GD, Nesterov y Adam. Puede, si le parece conveniente, implementar la versión estocástica.

    \item Incluir una penalización (regularización) en las $x$\'s: Esto es añadir a la función objetivo le término $\lambda \|x\|^2$, donde $\lambda$ es un parámetro que controla cuantas $\alpha$ se expresan.    

\end{itemize}


{\bf} La tareas se entregan en un "notebook" mediante classroom. Incluya la url de su imagen de prueba (leala de una dirección de internet).

\end{document}